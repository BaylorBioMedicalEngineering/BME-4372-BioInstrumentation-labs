\chapter{Raspberry Pi Ref}\label{chapter:RPiRef}

\section{Preparing Your Pi}

The Raspberry Pi can use a variety of OS's, but its basic one is a Linux distro.

\begin{description}
  \item[sudo] runs the command after it as another user, who is by default the super user, hence the name Super User DO.
  \item[apt-get] patches and upgrades the system and applications.
\end{description}

sudo apt-get update %get the latest version

sudo apt-get upgrade %upgrade what you have to what you updated

sudo apt-get install git %install git, great way to get lots of code

sudo apt-get install python-dev %install python development

sudo apt-get install python-rpi.gpi %install python raspberry pi gpio tools

\section{getting code}

git clone http://github.com/adafruit/Adafruit-Raspberry-Pi-Python-Code.git

cd Adafruit-Raspberry-Pi-Python-Code

ls

\section{coding}

For Raspberry Pi

The simplest way to ensure some cleanup code gets run is to wrap your while loop in a try/finally block.

If you specifically want to catch Ctrl+C but nothing else, you can use try/except to catch KeyboardInterrupt.

Finally, if you want a more robust way to ensure some cleanup gets done, take a look at lines 312 through 322 in this revision of my Procrastinator's Timeclock utility.

https://github.com/ssokolow/timeclock/blob/3bb9889...

First, it assigns the cleanup function to sys.exitfunc so it'll get called on clean exit, then it hooks all the common POSIX signals the kernel might send s ensure that they trigger a clean exit.

* SIGINT is sent by Ctrl+C, which INTerrupts the program.
* SIGTERM is sent by things like task managers politely asking your program to TERMinate itself.
* SIGHUP is sent by the kernel when your program loses the terminal it's attached to.
* SIGQUIT is sent by Ctrl+\, which asks the program to quit and dump core if core dumping is enabled.

(My code technically doesn't handle SIGQUIT properly, since it doesn't dump core, but I've never seen people using it properly.)

Finally, it also uses try/except to catch Ctrl+C, just to play it safe. (And it's still not perfect because PyGTK doesn't let the program attach a handler for "lost connection to X server")

If you want a version that's a bit more complicated, but also works on Windows, look at this revision:

https://github.com/ssokolow/timeclock/blob/7a50500...

(It checks that each signal exists before hooking it)
